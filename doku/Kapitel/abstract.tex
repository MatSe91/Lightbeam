\begin{abstract}
Diese Studienarbeit befasst sich mit der Entwicklung eines Puzzle-Spiels, welches Physik spielerisch lehrt. Hierbei werden wissenschaftliche Studien analysiert, die das Spiel als Lernmedium untersuchten. Basierend auf diesen Analysen wird ein Spielkonzept erstellt und programmiert. 

Einerseits wird sich mit der Thematik befasst, ob das Spiel in 2D oder 3D erstellt wird. Entsprechend dazu werden unterschiedliche Spiele-Engines betrachtet und analysiert, um die passendste zu ermitteln. 

Nach der Wahl / Vor der Wahl einer Engine wird sich mit den unterschiedlichen Bereichen der Physik befasst und eines als Grundlage des Spiels gewählt (Wir würden Unity als Engine verwenden).\\

Ein Spielkonzept könnte folgendermaßen aussehen:
\begin{itemize}
\item{Thematik:}  Licht
\item{Grundidee:} Ein Lichtstrahl soll durch Bewegungen auf ein Ziel gelenkt werden.
\item{Ziel:} Das Licht auf eine oder mehrere Knospen lenken, die zu einer Blume (o.ä.) heranwachsen und das Level abschließen
\item{optional:} Sammelobjekte vor dem Abschluss des Levels einsammeln, eine Punktesystem
\item{Einflüsse:} Manuell setzbare Spiegel (Reflexion), Flüssigkeiten(Brechung) und weitere Objekte aus der Lichtphysik 
\item{Features:} Je nach Anzahl der gesetzten Objekte und den gesammelten Objekten entsteht ein Ranking

    \begin{itemize}
        \item Schnittstelle: Erweiterbar mit weiteren physikalischen Bereiche
        \item Schnittstelle: die einzelnen Bereiche (z.B. Licht, Schwerkraft, Magnetismus) mit neuen Lv erweitern können
    \end{itemize}

\end{itemize}

\end{abstract}
\newpage