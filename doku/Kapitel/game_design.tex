\section{Game Design}

\subsection{Ziel des Spiels}
Das Ziel des Spiels ist es, einen Samen zu einem Weltenbaum zu entwickeln. 

\subsection{Erreichen des Ziels}
In einem Puzzle-Spiel müssen mehrere Level erfolgreich durchlaufen und Collectables gesammelt werden. Das erfolgt durch Steuerung eines Lichtstrahls, der auf das Ziel-Objekt des Levels treffen muss. Die Freischaltung der Level erfolgt durch Abschließen des vorherigen Levels und durch Besitzen einer definierten Anzahl Collectables. Zur vollständigen Entwicklung des Baumes werden 50 Level durchlaufen.

\subsection{Level Screen / Bereiche}
Das Konzept dieses Spiels beruht auf die Verwendung mehrerer Screens. Hierfür wird es einen Tutorialbereich geben, der die grundlegende Mechanik des Spiels erläutert. Dieses Tutorial und die ersten Level wird je 1 Screen bereit gestellt. In den weiterführenden Level steigt die Anzahl der Screens, die zum erreichen des Zielpunktes notwendig sind. Mehrere Screens zeichnen sich ähnlich wie ein Jump'n Run-Game aus, da sich die Kamera bzw. das Level automatisch bewegt. Bei diesem Spiel ist die Kamera und auch der Bereich fixiert. Erst beim erreichen des Lichtstrahls auf einem Checkpoint und dessen Aktivierung, verschiebt sich die Kamera zum nächsten Bereich des Levels. Eine Aufsplittung der Level und wie viele Bereiche diese haben sind in Tabelle \ref{screenLevel} aufgeführt.

\begin{table}[H]
\centering
\begin{tabular}{c|c}
Levelbereich & Anzahl der Screens / Bereiche \\ \hline
1 - 10  & 1 Screen \\ 
11 - 30 & 2 Screens \\
31 - 50 & 3 Screens      
\end{tabular}
\caption{Anzahl Screens in Levelbereiche}
\label{screenLevel}
\end{table}

%%%%%%%%%%%%%%%%%%%%%%%%%%%%% BEGIN OLD
Hierbei werden mehrere Level durchlaufen. Neue Level werden durch den Abschluss des vorherigen Levels und  eine definierte Anzahl an Collectables freigeschaltet. Im Durchschnitt sind 3 Coll. gedacht. Bei mehreren Screens können die Coll. variieren. Insgesamt sind 50 Lvl für ein Weltenbaum angedacht. Es werden maximal 3 Screens pro Lvl verwendet.
Collectables werden voraussichtlich als Wassertropfen dargestellt. Bei abschließen des Lvls / Bereichs fallen die Wassertropfen auf Pflanzen, die erblühen und ggf. die Map erhellen.

Collectables notwendig für nächstes Lv
\begin{description}
\item [Lv 1 - 5] 0\%, 3C je Lv 
\item [Lv 6 - 10] 30\%, 3C je Lv
\item [Lv 11 - 30] 50\%
\item [Lv 31 - 45] 70\%
\item [Lv 46 - 50] 70\%   (von der bisherigen Gesamtanzahl)
\end{description}
%%%%%%%%%%%%%%%%%%%%%%%%%%%%%%% END OLD

\subsection{Design des Spiels}
Das Spiel ist in einem schlichten, minimalistischen Design gehalten. Zu Beginn sind nur wenige Farben erkennbar, diese sind unter anderem schwarz, weiß, verschiedene Graustufen und Grüntöne. Durch Einsammeln der Collectables und Aktivierung des Checkpoint / Zielpunktes werden in einem definierten Bereich alle Assets farblich gestaltet. Dabei nehmen die Assets ihre ursprüngliche Farbe an, welche sie von Beginn an besaßen. Werden alle Collectables in einem Bereich eingesammelt werden auch alle Assets im gesamten Bereich farblich dargestellt.

Zudem ist der Stil des Spiels im Comic-Style gehalten. Die Aufbereitung der einzelnen Elemente (Wände, Pflanzen etc.) können allerdings alle möglichen Asset-Arten beinhalten. Die Kombination der unterschiedlichen Assets erzeugen den gewünschten Comic-Look.

\subsection{Wie wird der Checkpoint aktiviert / verwendet?}
Der Lichtstrahl muss den Ch-P berühren. Dazu muss der User den Point anklicken um in den nächsten Screen zu gelangen. Dadurch wird ein zufälliges aktivieren des Pointes verhindert.



\subsection{Spielelemente}
\begin{description}

\item [$\circledmark\quad$Spiegel] \hfill \\
Ein Spiegel ist ein Game Objekt, welches den Lichtstrahl reflektiert. Hierbei wird die korrekte Reflexion nach physikalischen Gesetzen verwendet. 

\begin{description}
    \item [$\circledmark\quad$rotierender Spiegel] \hfill \\
    Ein Spiegel welcher in einem Winkel von 140 Grad rotiert werden kann (Entgegengesetzt zur Wand).
    
    \item [$\circledmark\quad$Spiegel mit Farbeinschränkungen] \hfill \\
    Ein Spiegel, welcher nur bestimmte Lichter reflektiert.
    
    \item [$\circledmark\quad$beweglicher Spiegel] \hfill \\
    Ein Spiegel welcher auf einer Horizontalen/Vertikalen Bahn durch wischen bewegt werden kann.
    
    \item [$\circledmark\quad$feste Spiegel] \hfill \\
    Spiegel die keine Interaktionsmöglichkeiten besitzen.
    
\end{description}

\item [$\circledmark\quad$Prisma] \hfill \\
Wird als Kristall/Edelstein welcher Bunt schimmert(Farbe wechselt) dargestellt. Bei eintreffen des Lichtstrahls wird das Licht gebrochen und in die einzelnen Spektralfarben aufgeteilt. Das Prisma ist ein festes Objekt mit welchem keine weiteren Interaktionsmöglichkeiten bestehen.

\item [$\circledmark\quad$Wasserfall] \hfill \\
Wasserfälle funktionieren ähnlich wie Prismen, diese teilen das Licht durch das Wasser  in einen Regenbogen. Der Lichtstrahl muss im richtigen Winkel im Wasser auftreten um durch den Bogen die Checkpoints oder Türmechanismen zu aktivieren. Die Wasserfälle sollten animiert sein.

\item [$\circledmark\quad$Türmechanismus] \hfill \\
Türschalter werden als kleine Solarzellen dargestellt, welche Energie aufladen und nach kürzester Zeit ein Aufzugstor öffnen. Diese können aus einem oder zwei Mechanismen bestehen um ein Tor zu öffnen. Die Tore bestehen aus Stein/Holz/Blätter-platten die Textur ist Abhängigkeit der Level auszuwählen.

\item [Flüssigkeiten] (Optional) \hfill \\
Hierbei wird speziell auf das Wasser eingegangen. Dieses soll in der Seitenperspektive dargestellt werden. (Wahrscheinlich nur in der Höhle anwendbar) Der auf das Wasser treffende Lichtstrahl wird entsprechend physikalischer Bedingungen gebrochen. Beim austreten des Strahl aus dem Wasser werden ebenfalls physikalische Begebenheiten beachtet.

\item [$\circledmark\quad$Hintergründe] \hfill \\
Hintergründe dienen der Umgebungsgestaltung und den Eindruck einer 3D Umgebung. Diese bestehen aus Wänden, Felsen, Bergen, Bäumen, Pflanzen, Gräsern und Partikeleffekte. Der Hintergrund wird in normalen Farbe dargestellt.

\item [$\circledmark\quad$Vordergrund] \hfill \\
Vordergründe dienen der Umgebungsgestaltung und den Eindruck einer 3D Umgebung. Dieser besteht größtenteils aus Wänden , Pflanzen, Steinen, Felsen und Gräsern. Die Darstellung ist bis zum Aufdecken der Karte in minimalistischen Farben zu halten. (Farben mit wenig Intensität). Nach einsammeln aller Collectables wird die Karte als aufgedeckt behandelt. 

\item [$\circledmark\quad$Lichtstrahl] \hfill \\
Nachdem auswählen der jeweiligen Lichtquelle/Reflexionspunkt lässt der Lichtstrahl durch wischen auf dem Bildschirm in eine Richtung lenken.
Strahlt bis Ende des Bildschirms oder bis er auf ein Objekt trifft
Reflektiert beim Eintreffen auf ein Spiegelähnliches Objekt
Bricht in verschiedene Farbstrahlen beim Eintreffen auf ein Prisma.
Beim durchdringen von Wasserfällen verhält sich der Lichtstrahl  wie beim Prisma. und eine Animations für ein Regenbogen wird ausgeführt
Bei Blenden von Lebewesen werden Animation durchgeführt
Lichtstrahl beinhaltet Partikeleffekte für die optische Aufwertung
Der Lichtstrahl ist weiß

\item [Startpunkt] \hfill \\
Der Startpunkt eines jeden Levels ist eine Sonnenblume. Diese ist zunächst als einfache Pflanze zu sehen und wird durch natürliche Sonnenstrahlen beleuchtet. Durch anklicken der Pflanze wird eine Animation eingeleitet, welche die Sonnenblume erblühen lässt. Die Animation soll nicht länger als 1-2 Sekunden dauern, da sie kein direkten Bestandteil des Spielvergnügens darstellt. Ist sie vollends aufgeblüht erzeugt der Blütenkopf von selbst aus ein Lichtstrahl. Dieser Lichtstrahl dient als Spielmedium. 

Die Steuerung des Lichtstrahls erfolgt durch einmaliges antippen des Blütenkopfes. Dadurch lässt sich der Kopf um 360 Grad drehen, entsprechend dazu verändert sich ebenfalls die Richtung des Lichtstrahls.

\item [$\circledmark\quad$Zielpunkt] \hfill \\
Der Zielpunkt ist der Baum  selbst je nach Zustand des Baums/Fortschritt des Spiels (Samen/Sprössling/Baum). 
Sobald der Zielpunkt angestrahlt wird und das Rätsel gelöst ist kann durch ein Klick auf den Zielpunkt das Level beendet werden. Danach sieht man den Endscreen des Levels.
\item [Checkpoint]\hfill \\

\item [$\circledmark\quad$Pflanzen] \hfill \\
Pflanzen werden als Vordergrund/Hintergrund eingesetzt um die Umgebung zu gestalten. Diese werden in Schwarzweißer Optik und möglichst mit leichten Animationen dargestellt. Falls alle Collectables eingesammelt wurden, werden diese im entsprechenden Bereich nun in Farbe angezeigt.

\item [$\circledmark\quad$Wände] \hfill \\
Die Wände des Spiels werden durch unterschiedliche Assets begrenzt. Hierbei dienen Felsen, Bäume, Sträucher, Gräser, Lianen, Steine, Holz, Pflanzen. Sobald der Lichtstrahl auf eines dieser Elemente trifft, wird eine Animation für das Lichtende abgespielt. Der Lichtstrahl durchdringt diese Elemente in der Regel nicht, es sei denn sie befinden sich im Hintergrund / Vordergrund.

\item [Felsen]\hfill \\

\item [$\circledmark\quad$Hügel] \hfill \\
Berge und Hügel dienen zur Gestaltung des Hintergrunds.

\item [$\circledmark\quad$Collectables] \hfill \\
Hierbei handelt es sich um Wassertropfen. Diese können eingesammelt werden, indem der Lichtstrahl über diese gesteuert werden. Damit die Coll. eingesammelt werden, muss der Lichtstrahl zum Abschluss des Lv/Bereichs noch immer auf die Coll. verweilen, deshalb müssen diese so platziert werden dass auch bei einem Durchgang alle eingesammelt werden können.

\item [$\circledmark\quad$Animation] \hfill \\
Animationen werden eingesetzt um das Feedback einer Interaktion des Spieler zu visualisieren und den optischen Eindruck der Umgebung zu verbessern.

\item [$\circledmark\quad$Tierchen] \hfill \\
Tierchen dienen als Funfact und sorgen für eine authentische Umgebung. Optional werden Eastereggs durch kleine Animationen der Tierchen eingebaut falls diese vom Lichstrahl erwischt werden. ( Maulwurf holt Sonnenbrille, Spinne verkriecht sich)

\item [$\circledmark\quad$Weltenbaum] \hfill \\
Der Weltenbaum dient als Levelauswahlmenü und wird je nach Anzahl der freigeschalteten Level größer und umfangreicher gestaltet. Folgende Stadien des Baumes sind möglich Samen/Knospe/Strauch/Baum etc. Das Ziel des Spiels ist es den Baum zum wachsen zu bringen und ihn in seine vollste Pracht zu bringen.

\item [Hauptmenü] \hfill \\
siehe Zeichnung

\item [Optionsmenü] \hfill \\
siehe Zeichnung (Play und Optionen Buttons)

\item [Ansicht der Collectables] \hfill \\
Siehe Zeichnung 

\item [Abschlussdarstellung] \hfill \\
Siehe Zeichnung
Es ist möglich das Level Neuzustarten, das nächste Level zu starten, in die Levelansicht zu wechseln und/oder das gerade gespielte Level komplett zu betrachten. 

\item [Horizontale Verwendung] \hfill \\
Das Spiel wird nur in der Horizontalen Perspektive des Geräts dargestellt.

\item [Automatische Speicherung] \hfill \\
Das Spiel soll nach jedem Level automatisch gespeichert werden. Dafür wird lokal eine Speicherdatei erzeugt, die im Game Ordner versteckt angelegt wird. Die bereits vorhandene Optional wäre den Spielstand in die Googleplay Cloud zu laden.

\end{description}


\subsection{Setting}
\begin{description}
\item [Höhle]
\item [Erdoberfläche/Wiese]
\item [Wald]
\item [Baumkrone]
\end{description}

\subsection{optional}
\begin{description}
\item [Bauminneres]
\item [Himmel]
\end{description}
\newpage



%% Der Kreis mit dem Haken in der Mitte
%%  $\circledmark\quad$