\section{TODOs}

\subsection{Zwischenstand}
\begin{itemize}
\item Ausarbeitung Prototyp Level (vllt erste Konzepte für die ersten LV)
\item Ausarbeitung Assets (Teil 1)
\item Game Design Dokument
\item Komponenten anhand Gamedesign und Gamification Richtlinien erläutern (Genre, Direktes Feedback, Belohnungssysteme usw.)
\item Ausarbeitung Analyse ob Spiele lehren + Text verfassen
\item Ausarbeitung Analyse der versch. Game Engines (Vor-/ Nachteile) + Wahl welcher Engine + Text verfassen

\item Einarbeitung in Unity + Sichtung der notwendigen Physik skripte
\item Ermittlung der benötigten Assets (Grundlage)
\item Ermittlung der benötigten Effekte (wie Partikel)
\item Animationen definieren
\item Ermittlung benötigter physikalischer Formeln (Gegebenheiten)
\item Einarbeitung dokumentieren



\item Abgabe Anfang - Mitte Januar
\item 80\% des wissenschaftlichen Bereiches sollte abgehakt sein
\item auf jeden Fall God of Light mit aufnehmen als Inspiration
\item Analyse bzw. auch das Auseinandersetzen mit Physik und Game Engine anfangen Überarbeiten
\item $\rightarrow$ da kann man auch noch wissenschaftlichen Kran rausholen aus der Doku
\item ich (Matze) schätzt das wir ca 20-30\% wissenschaftlichen Teil erst erledigt haben
\item
\end{itemize}

\subsection{15.10.2015}
\begin{itemize}
\item \circledmark\quad  Studien bzgl Spielen (Text verfassen) $\rightarrow$ Matze
\item \circledmark\quad  Engines vergleichen (Engines verfassen)$\rightarrow$ Roman
\item \circledmark\quad  Spieldesign (Gutes Design/NoGos) $\rightarrow$ beide
\item \circledmark\quad  Assets in Unity $\rightarrow$ Matze
\item \circledmark\quad Fragen wegen Budget $\rightarrow$ Matze $\rightarrow$ Muss individuell beim Freudenmann geprüft werden
\end{itemize}

\subsection{22.10.2015}
\begin{itemize}
\item \circledmark\quad Game mechanics besprechen (Teils) $\rightarrow$ zusammen
\item \circledmark\quad Bücher weiter lesen und ggf. Passagen rausschreiben $\rightarrow$ beide
\item \circledmark\quad nebenher mit Unity befassen $\rightarrow$ beide
\item \circledmark\quad Game Elemente besprechen und aufschreiben (Teil 1) $\rightarrow$ zusammen
\end{itemize}

\subsection{29.10.2015}
\begin{itemize}
\item \circledmark\quad  Zeitplan erstellen (vorerst bis Ende Dezember)
\item \circledmark\quad  Bücher auslesen + rausschreiben $\rightarrow$ beide
\item Unity mehr einarbeiten
\end{itemize}

\subsection{05.11.2015}
\begin{itemize}
\item \circledmark\quad Mögliche Gliederungspunkte / Überschriften für Dokument bestimmen $\rightarrow$ beide
\item siehe vom 29.10. TODOs müssen noch durchgeführt werden
\end{itemize}

\subsection{12.11.2015}
\begin{itemize}
\item entfallen
\item Definition der Game-Elemente + Aussehen, Funktion, Interaktion $\rightarrow$ beide

\end{itemize}

\subsection{19.11.2015}
\begin{itemize}
\item Game Mechanics zu den Game Elementen genauer definieren
\item \circledmark\quad  1. Version Analysetext zu Spiele und Lernen $\rightarrow$ Matze
\item \circledmark\quad  1. Version Game Engines (Vor- Nachteile) $\rightarrow$ Roman
\item Game Design (Dokument) überarbeiten auf Vollständigkeit und Logik etc.
\end{itemize}

\subsection{26.11.2015}
\begin{itemize}
\item Assets Liste auswählen
\item FB Text für Assets schreiben
\item \circledmark\quad Texte in Latex konvertieren 
\item \circledmark\quad  Meilenstein Sonntag $\rightarrow$  unsere Texte \bfseries{müssen} fertig werden!!!!!  
\item \circledmark\quad  Emily kontaktieren
\end{itemize}


\subsection{03.12.2015}
\begin{itemize}
\item Roman fragt nach entgültigen Abgabetermin Zwischenstand
\item Mikael Gustafsson (asset publisher mikaelgustaf@gmail.com) anschreiben wegen kostenloser assets
\item Grafiker gesucht text in FB und Foren nachfragen
\item 
\end{itemize}

\subsection{Optimierungen}
\begin{itemize}
\item Gamesdesign Dokument strukturieren in Optisch/Funktion(interaktiv) (Spielelemente)
\item Genres und Grafikstil erklären (warum diese Entscheidung)
\item 
\end{itemize}

\subsection{10.12.2015}
\begin{itemize}
\item 
\item 
\item 
\item 
\end{itemize}

Abgabe Zwischenstand: 12.12.2015

%%%%%%%%%%%%%%%%%%%%%%%%%%%%%%%%%%%%%%%%%%%%%%%
%% Wenn man ein TODO abgeschlossen hat        %
%% bitte \circledmark\quad dahinter schreiben %
%% \circledmark\quad                          %
%%%%%%%%%%%%%%%%%%%%%%%%%%%%%%%%%%%%%%%%%%%%%%%

%%%%%%%%%%%%%%%%%%%%%%%%%%%%%%%%%%%%%%%%
%% Rechtspfeil                         %
%% $\rightarrow$                       %
%%%%%%%%%%%%%%%%%%%%%%%%%%%%%%%%%%%%%%%%