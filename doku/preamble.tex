%% Codierung
\usepackage[utf8]{inputenc}
%% Sprache
\usepackage[ngerman]{babel}
%% Anpassung der Seitenränder
\usepackage[a4paper,lmargin={2cm},rmargin={2cm},tmargin={2.0cm},bmargin = {2.5cm}]{geometry}
%% zum einbinden von Bildern
\usepackage{graphicx} 

\newcommand\circledmark[1][green]{%
  \ooalign{%
    \hidewidth
    \kern0.65ex\raisebox{-0.9ex}{\scalebox{3}{\textcolor{#1}{\textbullet}}}
    \hidewidth\cr
    $\checkmark$\cr
  }%
}

\newcommand\TODO[1]{\textcolor{red}{#1}}

\usepackage{float}
%% und zum positionieren der Bilder
\usepackage[export]{adjustbox}
%% Tabellen
\usepackage{array}
%% Einfärbung von Text
\usepackage{color} 
\usepackage{enumitem}
%% findet Verwendung im Inhaltsverzeichnis
\usepackage[hyphens]{url}
%% versteckt die Hyperref-Links als normalen Text
%% Informationen werden in PDF-Eigenschaften geschrieben
\usepackage[%
hidelinks,%
  pdftitle={Entwicklung eines Physikspiels, welches Physik spielerisch lehrt, auf Basis einer Engine},%
  pdfauthor={Roman Jung und Matthias Seyfarth},%
  pdfsubject={Entwicklung eines Physikspiels},%
  pdfproducer={PDFLaTeX},%
  pdfkeywords={Game; Spiel; lernen; lehren; Counter Strike; Engine; Unity; Studienarbeit; DHBW; Karlsruhe; Informatik}]{hyperref}
%% ermöglicht Querformat
\usepackage{pdflscape}
%% Quellcodelistings
\usepackage{listings}
\lstset{basicstyle=\footnotesize\ttfamily,breaklines=false} %monospace in listings
%Weil latex Kacke ist
\lstset{literate=%
  {Ö}{{\"O}}1
  {Ä}{{\"A}}1
  {Ü}{{\"U}}1
  {ß}{{\ss}}2
  {ü}{{\"u}}1
  {ä}{{\"a}}1
  {ö}{{\"o}}1
}
%für Suche von Umlauten
\usepackage[T1]{fontenc} 
%% ermöglicht den Haken hinter den TODOs
\usepackage{amssymb}
%% für das Abkürzungsverzeichnis
\usepackage{acronym}
%% dadurch kann man \degree (°) anwenden
\usepackage{gensymb}